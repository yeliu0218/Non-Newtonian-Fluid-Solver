\documentclass{article}

%%  Latex generated from POD in document /home/semar/pelican/CURRENT/tools/pel/pel.pl
%%  Using the perl module Pod::LaTeX
%%  Converted on Wed Mar 17 19:34:48 2010


\usepackage{makeidx}
\makeindex


\begin{document}

\tableofcontents

\clearpage
\section{pel\label{pel}\index{pel}}


Management of PELICANS-based applications

\subsection*{SYNOPSIS\label{pel_SYNOPSIS}\index{pel!SYNOPSIS}}


pel [-help$|$-man]



pel [options...] command [command-options-and-arguments]



command is either :

\begin{verbatim}
  depend         generation of makefiles
  build          generation of executables, objects and libraries
  run            execution of an application
  test           comparison between runs (for regression testing)
  time           management of a CPU time database
  arch           display of the available compiler architecture
  predoc         possible preparation before using "peldoc"
  newclass       text files creation for a new class
  cmp            requests forward to the PELICANS-based application "pelcmp"
  cvgce          extraction of error norms from "save_*.gene" files
\end{verbatim}


[command-options-and-arguments] : depends on the chosen command.

\subsection*{DESCRIPTION\label{pel_DESCRIPTION}\index{pel!DESCRIPTION}}


\texttt{pel} is a collection of utilities devoted to the management
of PELICANS-based applications.

\subsection*{OPTIONS\label{pel_OPTIONS}\index{pel!OPTIONS}}


The following options are those of \texttt{pel} itself. See the
manpages of the various commands for their specific options.

\begin{description}

\item[\textbf{-h, -help}] \mbox{}

Print a brief help message and exit.


\item[\textbf{-m, -man}] \mbox{}

Print the manual page and exit.


\item[\textbf{-v, -verbose}] \mbox{}

Execute verbosely.

\end{description}
\clearpage
\section{depend\label{depend}\index{depend}}


Generation of makefiles for PELICANS-based applications

\subsection*{SYNOPSIS\label{depend_SYNOPSIS}\index{depend!SYNOPSIS}}


pel depend [-help$|$-man]



pel depend [options...] arguments...



pel depend [-l lib$|$dir] opt \emph{bindir} \emph{sources} [-precomp directory]



pel depend -vcproj $<$project.vcproj$>$ \emph{sources} [-precomp directory]

\subsection*{DESCRIPTION\label{depend_DESCRIPTION}\index{depend!DESCRIPTION}}


The task of compiling a PELICANS-based application is highly
simplified by using the GNU \texttt{make} utility. The aim
of \texttt{pel depend} is to write a suitable makefile
that describes the relationships among
files in the considered application and provides
commands for compiling each file and linking with the appropriate
PELICANS library. Once this makefile exists, \texttt{pel build}
can be used to build the desired executable file.



\texttt{pel depend} has been specially designed to handle
sources located in multiple directories, and to handle
multiple compilers on the same file system. Moreover
the generated makefile includes special targets and commands
to determine or re-determine the dependencies between the files if necessary.



The object oriented methodology of PELICANS strongly relies on
built-in assertions : preconditions, postconditions, invariants and
checks which are implemented using respectively the four
macros : \texttt{PEL\_CHECK\_PRE}, \texttt{PEL\_CHECK\_POST}, \texttt{PEL\_CHECK\_INV}
and \texttt{PEL\_CHECK}.
These assertions can be enabled/desabled at translation time and by
using command-line switches. Some options and
arguments of \texttt{pel depend} and \texttt{pel run} are specially devoted to
this task.

\subsection*{OPTIONS\label{depend_OPTIONS}\index{depend!OPTIONS}}
\begin{description}

\item[\textbf{-h, -help}] \mbox{}

Print a brief help message and exit.


\item[\textbf{-m, -man}] \mbox{}

Print the manual page and exit.


\item[\textbf{-v, -verbose}] \mbox{}

Execute verbosely.


\item[\textbf{-D} name] \mbox{}

In the generated makefile, predefine \texttt{name} as a macro,
with definition 1.
This option may be used any number of times.


\item[\textbf{-D} name=value] \mbox{}

In the generated makefile, predefine \texttt{name} as a macro,
with definition \texttt{value}.
This option may be used any number of times.


\item[\textbf{-compiler} comp] \mbox{}

In the generated makefile, use the
the compiler denoted by \texttt{comp} for all
tasks of preprocessing, compilation, assembly
and linking. Default is: \texttt{gcc}.


\item[\textbf{-makefile} makefile] \mbox{}

Set the name of generated makefile used for all
tasks of compilation, assembly and linking.
Default is: \emph{bindir}/Makefile.


\item[\textbf{-l} \emph{lib$|$dir}] \mbox{}

Add archive file \emph{lib} to the list of files to link.
The commands of the generated makefile will invoke
the linker with "-l\emph{lib}" on the command line.



When the argument is a directory, add all objects files found in the \emph{dir} directory
to the list of files to link.



This option may be used any number of times.


\item[\textbf{-precomp} \emph{dir}] \mbox{}

Examine directory to use precompiled objects if any.


\item[\textbf{-path} \emph{searchdir}] \mbox{}

Add directory \emph{searchdir} to the list of paths that the linker
will search for archive libraries when invoked by the 
commands of the generated makefile. 
This option may be used any number of times.


\item[\textbf{-I} \emph{searchdir}] \mbox{}

Add directory \emph{searchdir} to the list of paths that the compiler will search
for header files when invoked by the commands of the generated
makefile. Note that any subdirectory of $<$sources$>$ that contains a header
file is automatically added to that list of paths.
This option may be used any number of times.


\item[\textbf{-profile}] \mbox{}

In the generated makefile, activate the profiling option
when invoking the compiler and linker.
This allows the profiling analysis of the application 
with tools such as gprof.


\item[\textbf{-coverage}] \mbox{}

In the generated makefile, activate the basic block coverage analysis option
when invoking the compiler and linker.
This allows the structural analysis of the application 
with tools such as tcov.


\item[\textbf{-inline $|$ -include}] \mbox{}

Decide how external makefiles (eg \emph{extra-Linux.mak}) taken from the PELICANS
repository are accounted for in the generated makefile.
If \texttt{-inline}, they are copied line by line  (default
behaviour). If \texttt{-include}, the \texttt{include} directive of gmake is used.


\item[\textbf{-mSTD}] \mbox{}

Add the directories of the PELICANS library "ExamplesOfApplication"
to the list of paths to be searched for sources.


\item[\textbf{-mPELICANS}] \mbox{}

Add the directories of the PELICANS repository (except
those of the library "ExamplesOfApplication") to the list
of paths to be searched for sources. This option is
reserved to the internal administration of PELICANS.


\item[\textbf{-vcproj project.vcproj}] \mbox{}

Update Visual C++ project by updating list of include directories
and list of files.

\end{description}
\subsection*{ARGUMENTS\label{depend_ARGUMENTS}\index{depend!ARGUMENTS}}
\begin{description}

\item[\textbf{opt}] \mbox{}

Decide the compilation level that will be set in the
generated makefile. This option influences on the one
hand the optimization used by the compiler when generating
the binary code and on the other hand the assertions that
will be evaluated. Allowed values for \texttt{opt}
are: \texttt{dbg}, \texttt{opt2}, \texttt{opt1}, \texttt{opt0}, \texttt{optpg} or \texttt{optcov}.

\begin{description}

\item[\textbf{dbg}] \mbox{}

The commands of the generated makefile will ask the compiling system
to generate a binary code prepared for debugging.


\item[\textbf{opt2},\textbf{opt1},\textbf{opt0}] \mbox{}

The commands of the generated makefile will ask the compiling system
to use an extensive set of optimization techniques
when generating the binary code.


\item[\textbf{dbg},\textbf{opt2}] \mbox{}

In the generated makefile, the commands invoking the compiler
will define the preprocessor name \texttt{LEVEL} as \texttt{2}. Thus,
when running the application,
the preconditions will be evaluated, and the postconditions, invariants
and checks will be possibly evaluated (depending on command-line switches,
see \texttt{pel run}).


\item[\textbf{opt1}] \mbox{}

In the generated makefile, the commands invoking the compiler
will define the preprocessor name \texttt{LEVEL} as \texttt{1}. Thus
the preconditions will be evaluated when running the application,
but the statement associated to
postconditions, invariants and checks will be removed during
preprocessing stage.


\item[\textbf{opt0}] \mbox{}

In the generated makefile, the commands invoking the compiler
will define the preprocessor name \texttt{LEVEL} as \texttt{0}. Thus
any statement associated to a precondition, a postcondition, an
invariant or a check will be removed during the
preprocessing stage.


\item[\textbf{optpg}] \mbox{}

Same as \textbf{opt0} combined with \textbf{-profile}. Sets the most agressive optimisation level and sets the profiling option.


\item[\textbf{optcov}] \mbox{}

Same as \textbf{-coverage}. Neither optimisation options nor debug information are generated by the compiling system.

\end{description}

\item[\textbf{ \emph{bindir} }] \mbox{}

Any file produced (objects, libraries, executables, dependency files ...)
will be located in \emph{bindir}. By default, the generated makefile,
called \emph{Makefile}, is created in the directory \emph{bindir}.


\item[\textbf{ \emph{sources} }] \mbox{}

A list of directories and source files.
\texttt{pel depend} will add the given source files and will add the source files found in the given directories.
Any header file or source file found in
a subdirectory of \emph{sources} is considered to be part of the application.
Header files are those having a \emph{.h} or \emph{.hh} extension whereas source
files are those having a \emph{.cpp}, \emph{.cc}, \emph{.c}, \emph{.F} or \emph{.f} extension.

\end{description}
\subsection*{EXAMPLES\label{depend_EXAMPLES}\index{depend!EXAMPLES}}
\begin{description}

\item[\texttt{pel depend -l pel1 dbg lib .}] \mbox{}

The current application is made of all the header and source files
located in any subdirectory of the working directory. 
The generated \emph{Makefile} will be created
in the directory \emph{lib}. Further compilations will be performed with the
\emph{dbg} compilation level, and linking will be performed with the library
\emph{libpel1.so}.


\item[\texttt{pel depend -l pel0 -I hea -compiler gcc opt1 bin src}] \mbox{}

The current application is made of all the header and source files
located in any subdirectory of \emph{src} . The generated \emph{Makefile} will be
created in the directory \emph{bin}. Further compilations will be performed
by \texttt{gcc} with the \texttt{opt1} compilation level, and linking will be performed
with the library \emph{libpel0.so}. The current application probably uses header
files that are not in the directory \emph{src} and that are not PELICANS header
files since the options \texttt{-I hea} is used.


\item[\texttt{pel depend opt1 bin src pelsrc /home/users/algo.cc -precomp lib}] \mbox{}

Compile the single source file /home/users/algo.cc and the source directories src and pelsrc using precompiled objects in lib.

\end{description}
\subsection*{ENVIRONMENT\label{depend_ENVIRONMENT}\index{depend!ENVIRONMENT}}


It is possible to store arguments and options, overwritable by the command
line arguments, in the environment variable PELDEPEND.

\clearpage
\section{build\label{build}\index{build}}


Generation of executables, objects and libraries related to PELICANS-based applications

\subsection*{SYNOPSIS\label{build_SYNOPSIS}\index{build!SYNOPSIS}}


pel build [-help$|$-man]



pel build [options...] -exe \emph{bindir}



pel build [options...] -object \emph{filename.o} \emph{bindir}



pel build [options...] -archive \emph{archname.so} \emph{bindir}

\subsection*{DESCRIPTION\label{build_DESCRIPTION}\index{build!DESCRIPTION}}


The generation of executables, objects and libraries associated to
a particular PELICANS-based application can be simply performed by using
the GNU \texttt{make} utility. The first step consists in building
a suitable makefile with the \texttt{pel depend} utility. The second
step consists in running GNU \texttt{make}  with suitable arguments,
a task whose responsiblity is assigned to \texttt{pel build}.



Essentially, GNU \texttt{make} is invoked with the
following instructions:

\begin{enumerate}

\item 1.

read the makefile called \emph{Makefile}, located in the directory \emph{bindir}
(unless otherwise specified with the \texttt{-makefile} options);


\item 2.

update the target determined by the mutually exclusive instructions
\texttt{-exe,-object,-archive};


\item 3.

use a specific set of make options that are determined by
the calling options of \texttt{pel build}.

\end{enumerate}
\subsection*{OPTIONS\label{build_OPTIONS}\index{build!OPTIONS}}
\begin{description}

\item[\textbf{-h, -help}] \mbox{}

Print a brief help message and exit.


\item[\textbf{-m, -man}] \mbox{}

Print the manual page and exit.


\item[\textbf{-v, -verbose}] \mbox{}

Execute verbosely.


\item[\textbf{-e, -E, -exe}] \mbox{}

Ask \texttt{make} to
update the target defined by 
the executable of the PELICANS-based application.


\item[\textbf{-o, -O, -object} \emph{filename.o}] \mbox{}

Ask \texttt{make} to update the target
defined by the module object \emph{filename.o}
(typically, \texttt{filename} is the name of one of
the classes making up
the considered application).


\item[\textbf{-a, -A, -archive} \emph{archname.ext}] \mbox{}

Ask \texttt{make} to update the target
defined by the archive \emph{archname.ext}
(the associated command depends on
the extension \emph{.ext}).


\item[\textbf{-link\_mode} link\_mode] \mbox{}

Set the linker:
    link\_mode=cc : use the c++ linker (default)
    link\_mode=c  : use the c linker
    link\_mode=f  : use the fortran linker


\item[\textbf{-make} makename] \mbox{}

Use the command of name \texttt{makename} as 
the make command (default: \texttt{make}).
This option is typically used for systems
on which the GNU \texttt{make} utility is
called \texttt{gmake}.


\item[\textbf{-makefile} makefile] \mbox{}

Set the makefile used for all tasks of compilation,
assembly and linking. Default is: \emph{bindir}/Makefile.


\item[\textbf{-with} EXTlist] \mbox{}

Notify that the current application DOES require the
packages of \texttt{EXTlist} so that the archives
associated to those packages
should be added to the list of files to link.
\texttt{EXTlist} is a comma separated list denoting external APIs
that might be used by some components of PELICANS.
This option is meaningful only for the targets
of the options \texttt{-exe} and \texttt{-archive}.
Note that \texttt{pel depend} defines some defaults in the
generated makefile for all possible external libraries.


\item[\textbf{-without} EXTlist] \mbox{}

Notify that the current application DOES NOT require the
packages of \texttt{EXTlist} so that the archives
associated to those packages
should NOT be added to the list of files to link.
\texttt{EXTlist} is a comma separated list denoting external APIs
that might be used by some components of PELICANS.
This option is meaningful only for the targets
of the options \texttt{-exe} and \texttt{-archive}.
Note that \texttt{pel depend} defines some defaults in the
generated makefile for all possible external libraries.

\end{description}
\subsection*{ARGUMENT\label{build_ARGUMENT}\index{build!ARGUMENT}}
\begin{description}

\item[\textbf{ \emph{bindir} }] \mbox{}

Directory containing the makefile generated by \texttt{pel depend},
called \emph{Makefile}. Any file produced by the execution
of \texttt{pel build} (that is: when updating a target of \emph{Makefile})
will be created in that directory (objects, libraries, executable,
dependency files ...). The executable will be called \emph{exe}.

\end{description}
\subsection*{EXAMPLES\label{build_EXAMPLES}\index{build!EXAMPLES}}
\begin{description}

\item[\texttt{pel depend -l pel1 dbg bin/dbg .}] \mbox{}
\item[\texttt{pel build -exe bin/dbg}] \mbox{}

The current application is made of all the header and source files
located in any subdirectory of the working directory. The
generated \emph{Makefile}, the executable \emph{exe} and all the files
created during the compiling process will be located in the
subdirectory \emph{bin/dbg} of the working directory.


\item[\texttt{pel build -without petsc,opengl -exe lib}] \mbox{}

All the files generated for and during the compilation process
are located in the subdirectory lib of the working directory
(and possibly in the working directory itself).
Linking will be performed without the archives associated
to the PETSc and OpenGL libraries.


\item[\texttt{pel build -object myclass.o /usr/smith/appli/bin}] \mbox{}

Update the target \emph{myclass.o} of the makefile \emph{Makefile}
located in the directory \emph{/usr/smith/appli/bin}, that
is: build object file \emph{myclass.o} in that directory.

\end{description}
\subsection*{ENVIRONMENT\label{build_ENVIRONMENT}\index{build!ENVIRONMENT}}


It is possible to store arguments and options, overwritable by the command
line arguments, in the environment variable PELBUILD.

\clearpage
\section{run\label{run}\index{run}}


Execution of a PELICANS-based application

\subsection*{SYNOPSIS\label{run_SYNOPSIS}\index{run!SYNOPSIS}}


pel run [-help$|$-man]



pel run [options...] \emph{exe} \emph{data} \emph{resu}



pel run -build\_pattern \emph{filename} -R \emph{dir} \emph{exe} \emph{data} \emph{resu}



pel run -check\_pattern \emph{filename} \emph{exe} \emph{data} \emph{resu}

\subsection*{DESCRIPTION\label{run_DESCRIPTION}\index{run!DESCRIPTION}}


\texttt{pel run} executes a given PELICANS-based application with a
given data file, and copy all the output messages in a given file
(or set of files in parallel mode).



Execution is performed in the current directory unless the option
\texttt{-R} is specified. Other options are devoted to the customization
of the execution.

\subsection*{ARGUMENTS\label{run_ARGUMENTS}\index{run!ARGUMENTS}}
\begin{description}

\item[\textbf{ \emph{exe} }] \mbox{}

Name of the executable of the PELICANS-based application to run
(this executable has usually been built with a \texttt{pel build}
command).


\item[\textbf{ \emph{data} }] \mbox{}

Name of the file storing the PELICANS Hierarchical Data Structure
used as a data file for the PELICANS-based application to run.


\item[\textbf{ \emph{resu} }] \mbox{}

In sequential mode :
name of a file into which any message directed to the standard
error or to the standard output will be copied (this file will
be created or truncated).
In parallel mode : basename of the files into which any message
directed to the streams \texttt{PEL::out()} and \texttt{PEL::err()} will
be copied (these files will be created or truncated). The extension
of these files is the rank of their associated process.

\end{description}
\subsection*{OPTIONS\label{run_OPTIONS}\index{run!OPTIONS}}
\begin{description}

\item[\textbf{-h, -help}] \mbox{}

Print a brief help message and exit.


\item[\textbf{-m, -man}] \mbox{}

Print the manual page and exit.


\item[\textbf{-noverb}] \mbox{}

Deactivate the verbose option of the PELICANS-based application
(which is activated by default).


\item[\textbf{-R} \emph{dir}] \mbox{}

Instead of running in the current directory, run recursively in
all the subdirectories of \emph{dir} (including \emph{dir} itself)
that contain a file of name \emph{data}.


\item[\textbf{-Cpost}] \mbox{}

Activate the evaluation of the postconditions (only effective
for sources compiled with the '\texttt{opt2}' or '\texttt{dbg}' compilation level, 
see \texttt{pel depend} ).


\item[\textbf{-Call}] \mbox{}

Activate the evaluation of the postconditions, invariants and
checks (only effective
for sources compiled with the '\texttt{opt2}' or '\texttt{dbg}' compilation level,
see \texttt{pel depend} ).


\item[\textbf{-Xpetsc} \emph{option}] \mbox{}

Transfer \emph{option} to the PETSc external API.


\item[\textbf{-X} \emph{option}] \mbox{}

Transfer \emph{option} to external APIs.

\end{description}


To execute a PELICANS-based application in parallel on
multiple processors, \texttt{pel run} forwards its request
to \texttt{mpirun}.

\begin{description}

\item[\textbf{-np} n] \mbox{}

Specify the number of processors to run on
(call \texttt{mpirun} with \textbf{-np} n as an option).


\item[\textbf{-machinefile} \emph{file}] \mbox{}

Take the list of possible machines to run on from
the file \emph{file} (call \texttt{mpirun} with
\textbf{-machinefile} \emph{file} as an option).


\item[\textbf{-nolocal}] \mbox{}

Call \texttt{mpirun} with \textbf{-nolocal} as an option.

\end{description}


During the final stage of the execution of a PELICANS-based application,
the object to which \texttt{PEL\_Root::object()} refers is destroyed
(internally in the PELICANS framework), leading to
the destruction of all remaining instances of subclasses of \texttt{PEL\_Object}
whose owner is not the \texttt{NULL} object. But it is the reponsibility of
the developer of the PELICANS-based application to call the \texttt{destroy()}
method on behalf of any objects that he may have created with \texttt{NULL}
as owner. If such is not the case, some dynamically allocated memory
might not be released when the execution terminates, and an ad-hoc
message is printed by PELICANS. The following two options
help identifying the functions in which non-destroyed objects
have been created. They should be used during two successive runs:
the first run, with the \texttt{-Cobject} option identifies a non-destroyed
object whereas the second run, with the \texttt{-catch} option locates
the function in which it had been created.

\begin{description}

\item[\textbf{-Cobjects}] \mbox{}

Assign an identification number to each object that
has not been destroyed when the execution terminates,
and display all available information about them
(the identification number is an integer appearing
between brackets on top of each printed block).


\item[\textbf{-catch} nb] \mbox{}

Display a warning message at runtime when the
object whose identification number is \texttt{nb}
is created (necessarily with the \texttt{NULL} owner).

\end{description}


A PELICANS-based application that can be executed by \texttt{pel run}
requires a data file (whose name is the argument \emph{data}).
That file stores a Hierarchical Data System whose structure
is implicitely defined by the application itself (through
calls to the various member functions of \texttt{PEL\_ModuleExplorer}).
PELICANS offers the possibility to "learn" that structure
during an execution and to record this infered knowledge
in a called "pattern" file (option \texttt{-build\_pattern}).
This pattern file can be used in turn to check
the conformity of other data file with this structure
(option \texttt{-check\_pattern}).



These two options are mutually exclusive. The activation of
one of them inhibits the autocheck feature (see below).

\begin{description}

\item[\textbf{-build\_pattern} \emph{filename}] \mbox{}

During the run, extract the pattern of the hierarchical
data structure associated to \emph{data}, and add it to the
file called \emph{filename}
(which is created if it did not existed before, or extended otherwise).


\item[\textbf{-check\_pattern} \emph{filename}] \mbox{}

Prior to execution, check the conformity of \emph{data}
with the pattern file \emph{filename}.

\end{description}


Let us consider an execution associated to a given data file \emph{data}. If
there is no available pattern file against which the conformity
of \emph{data} can be checked, \texttt{pel run} offers an "autocheck"
mode which splits the execution in two steps: a run is first performed
and a pattern is extracted from \emph{data} (in a temporary file), then, after
completion of that run, \emph{data} is checked against the extracted
pattern. This mode allows finding unread parts in the data
file which may be caused by typing errors.

\begin{description}

\item[\textbf{-autocheck}] \mbox{}

Perform the run in the "autocheck" mode. This option is activated by default,
unless the options \textbf{-build\_pattern} or \textbf{-check\_pattern} are present.


\item[\textbf{-noautocheck}] \mbox{}

Deactivate the "autocheck" mode.

\end{description}
\subsection*{EXAMPLES\label{run_EXAMPLES}\index{run!EXAMPLES}}
\begin{description}

\item[\texttt{pel run  ../bin/exe data.pel resu}] \mbox{}

Run executable \emph{exe} located in the directory \emph{../bin/}
with the data file \emph{data.pel}, store
all outputs in the file \emph{resu} and perform an autocheck on \emph{data.pel}.


\item[\texttt{pel run -Cpost ../bin/exe data.pel resu}] \mbox{}

Same as before, with in addition the evaluation of all postconditions
of the member functions compiled with the '\texttt{dbg}' or '\texttt{opt2}'
compilation level.


\item[\texttt{pel run -np 3 -machinefile ms \$EXE0 data.pel resu}] \mbox{}

Launch the executable whose name is stored in the \texttt{EXEO} environment
variable on 3 processors, with the data file \emph{data.pel}.
The first process will be on the current machine, whereas the other ones will
be on the machines defined in the file \emph{ms}).
Outputs directed to \texttt{PEL::out()}
and \texttt{PEL::err()} will be copied in the files \emph{resu.0}, \emph{resu.1}
and \emph{resu.2} (first, second and third process).


\item[\texttt{pel run -np 3 -machinefile ms -nolocal \$EXE0 data.pel resu}] \mbox{}

Same as before, but the 3 processes will be launched on the machines defined
in the file \emph{ms}.


\item[\texttt{pel run -np 3 -machinefile ms -nolocal -Xpetsc -trace \$EXE0 data.pel resu}] \mbox{}

Same as before, with a transfer of the \emph{-trace} option to PETSc.


\item[\texttt{pel run -R Test bin/exe data.pel resu}] \mbox{}

If \texttt{EXE} denotes the file \emph{exe} located in the subdirectory
\emph{bin} of the working directory, this command is equivalent as
executing: $\:$\texttt{pel$\:$run$\:$EXE$\:$data.pel$\:$resu}$\:$
in all the subdirectories of \emph{Test} containing
a file \emph{data.pel}.


\item[\texttt{pel run -R -build\_pattern etc/pattern.pel bin/exe data.pel resu}] \mbox{}

Same as before, with in addition the learning and storage of the
requested structure of the data files in \emph{etc/pattern.pel}. 
No autocheck is performed.


\item[\texttt{pel run -check\_pattern ../etc/pattern.pel ../bin/exe data.pel resu}] \mbox{}

Check the conformance of \emph{data.pel} with \emph{../etc/pattern.pel}
and, if successful, run subsequently \emph{../bin/exe} with data file
\emph{data.pel} and store all outputs in the file \emph{resu}. No autocheck is
performed.

\end{description}
\clearpage
\section{test\label{test}\index{test}}


Comparison between runs of a PELICANS-based application (for regression testing)

\subsection*{SYNOPSIS\label{test_SYNOPSIS}\index{test!SYNOPSIS}}


pel test [-help$|$-man]



pel test [options...] \emph{exe} \emph{dirs}



pel test -build\_pattern \emph{filename} \emph{exe} \emph{dirs}



pel test -verify\_pattern \emph{filename} \emph{exe} \emph{dirs}



pel test -build\_then\_verify\_pattern \emph{filename} \emph{exe} \emph{dirs}

\subsection*{DESCRIPTION\label{test_DESCRIPTION}\index{test!DESCRIPTION}}


Comparing in details two runs of an application is important for
the sake of non regression or installation testing.



Given a hierarchy of directories containing reference runs
of a given application, \texttt{pel test} will perform the following actions:

\begin{enumerate}

\item 1.

create, in the
working directory, a subdirectory
in which that hierarchy is duplicated;


\item 2.

in all subdirectories of the duplicate hierarchy, run the application
with the associated reference data file (called \emph{data.pel});


\item 3.

compare the results of this run with the reference results.

\end{enumerate}


On completion, the conclusions of all these comparisons are recorded
in a report file located in the current directory.



The essentials of \texttt{pel test} tasks are forwarded to the PELICANS-based
application "peltest" (contained in the executable specified by the
\textbf{-peltest\_exe} option if any, or by the argument \emph{exe}).



Further details are given below.

\begin{description}

\item[.] \mbox{}

If a test failure is pronounced, the character sequence
"Test failed" will appear in the report file.


\item[.] \mbox{}

Runs are performed with the commands like:



sequential: \emph{exe} \emph{data.pel} -v [opts...] $>$ \emph{resu}



parallel:   \emph{mpirun} [mpi\_opts...] \emph{exe} \emph{data.pel} -v -o \emph{resu} [opts...]



where the options \texttt{opts...} and \texttt{mpi\_opts...} are determined
for all runs by the calling options of \texttt{pel test} (see below)
and for one specific run by a possible file \emph{config.pel} in
the reference directory of that run.



A file \emph{config.pel} containing:

\begin{verbatim}
  MODULE test_config
    run_options = vector( "...", "..." )
  END MODULE test_config
\end{verbatim}


leads to the addition in \texttt{opts...} of all items in the StringVector
(only for the run associated to the directory
containing the considered file \emph{config.pel}).



A file \emph{config.pel} containing:

\begin{verbatim}
  MODULE test_config
    mpi_options = vector( "...", "..." )
  END MODULE test_config
\end{verbatim}


switches to parallel execution and leads to the addition in \texttt{mpi\_opts...} of
all items in the StringVector (only for the run associated to the directory
containing the considered file \emph{config.pel}).
An additional optional StringVector data of keyword \texttt{mpi\_machinefile}
can be used to specify the list of possible machines to run on
(this data is written on a temporary machine file transmitted to \emph{mpirun}
via the option \texttt{-machinefile}).


\item[.] \mbox{}

The exit code is tested. The test failure is pronounced
if it is non zero, unless it exists a file \emph{config.pel}, stored in
the reference directory, containing:

\begin{verbatim}
  MODULE test_config
    failure_expected = true
  END MODULE test_config
\end{verbatim}


in which case success might be pronounced if exit code is non zero and
one or more produced files called \emph{expected.err*} are identical
(same name and same content) to those present in the reference directory.


\item[.] \mbox{}

The test failure is pronounced if the \emph{resu} file has not been produced.


\item[.] \mbox{}

All files that have been produced during the run (other than \emph{resu}) 
are compared (as described below) with the reference ones (that must exist).
This comparison might be avoided from some particular file of a particular
run if the associated reference directory stores a file \emph{config.pel}
containing:

\begin{verbatim}
  MODULE test_config
    files_to_ignore = vector( "...", "..." )
  END MODULE test_config
\end{verbatim}


The items of the StringVector correspond to files produced during the run for
which no comparison will be performed.


\item[.] \mbox{}

The comparison method between the reference and produced files
depends on the format of these files.



There are three "native" formats understood by PELICANS:
the format called GENE, for \texttt{TIC} postprocessing; the
format called PEL, for Hierarchical Data Structures with the PELICANS format;
the format called CSV, for comma separated values.



The files with format GENE, PEL or CSV are compared to the reference ones
with the PELICANS-based application "pelcmp" (contained in the
executable specified by the \textbf{-peltest\_exe} option if any, or by the
argument \emph{exe}). If they are not identical, the comparison results
are recorded in the report file (the test failure is not pronounced since
differences may be acceptable, depending on the use case).



The other files are compared line by line with the reference ones.
If they are not the same, the test failure is pronounced.


\item[.] \mbox{}

The format of a file, say \emph{save.zzz}, is determined as follows.
It can be specified via a configuration file \emph{config.pel} stored
in the reference directory:

\begin{verbatim}
  MODULE test_config
    MODULE PEL_Comparator
      MODULE xxx               // xxx is a non significant name
        filename = "save.zzz"
        format = "CSV"         // either "GENE", "PEL" of "CSV"
      END MODULE xxx
    END MODULE PEL_Comparator
  END MODULE test_config
\end{verbatim}


If such a specification is absent, the format is identified on the
basis of a motif appearing in the file name: \emph{.gene} gives the GENE
format, \emph{.pel} gives the PEL format, \emph{.csv} gives the CSV format.


\item[.] \mbox{}

The files with format GENE or PEL contain data identified by keywords.
The comparison might ignore some of these data if the associated
reference directory stores a file \emph{config.pel}
containing:

\begin{verbatim}
  MODULE test_config
    MODULE PEL_Comparator
      MODULE xxx               // xxx is a non significant name
        filename = "save.zzz"  // file with format GENE or PEL
        ignore_data = vector( "...", "..." )
      END MODULE xxx
    END MODULE PEL_Comparator
  END MODULE test_config
\end{verbatim}


The items of the StringVector correspond to keywords of data that
should be ignored during the comparison.


\item[.] \mbox{}

The floating point values contained in the reference and produced files
(with format GENE, PEL or CSV) are compared with PEL::double\_equality.
The last two arguments of
this member function are respectively called a\_dbl\_eps (a
kind of tolerance on relative errors) and a\_dbl\_min (a lower bound under which
values are undistinguishable from zero).



By default, a\_dbl\_eps and a\_dbl\_min are equal to zero (which means that
comparisons without any tolerance are performed).
They can be given other values either globally (for all runs)
using the \texttt{-dbl\_eps} and \texttt{-dbl\_min} options, or for a specific
run via a file \emph{config.pel} in the reference directory of that
run.



For instance, a file \emph{config.pel} containing:

\begin{verbatim}
  MODULE test_config
     MODULE PEL_Comparator
        MODULE xxx                       // xxx is a non significant name
           filename = "save.csv"
           MODULE double_comparison
              dbl_min = 1.e-10
              dbl_eps = 1.e-8
           END MODULE double_comparison
        END MODULE xxx
     END MODULE PEL_Comparator
  END MODULE test_config
\end{verbatim}


will set a\_dbl\_min=1.e-10 and a\_dbl\_eps=1.e-8 for comparisons between
the floating point values of the files \emph{save.csv}.



Note that the command line options \texttt{-dbl\_eps} and \texttt{-dbl\_min}
always overread the options stated in the files \emph{config.pel}.
Moreover the line option \texttt{-exact} can be used to ignore any
setting of a\_dbl\_min and a\_dbl\_max in the \emph{config.pel} files.

\end{description}


When the \texttt{-verify\_pattern} option is activated, the behavior
of \texttt{pel test} is slightly different: the only test performed
is the conformance of the reference data file \emph{data.pel} with
the given pattern file.

\subsection*{ARGUMENTS\label{test_ARGUMENTS}\index{test!ARGUMENTS}}
\begin{description}

\item[\textbf{exe}] \mbox{}

Name of the executable of the PELICANS-based application to run.


\item[\textbf{dirs}] \mbox{}

List of the directories defining the reference runs. Any subdirectory
of an item of \emph{dirs} containing a file \emph{data.pel} is considered
by \texttt{pel test} as a definition of a reference run whose data file
is \emph{data.pel}. This subdirectory must contain the reference
version of all the files produced when calling \emph{exe} with that
data file. It might also contain (see above) a file called \emph{config.pel}
and, more rarely, files called \emph{expected.err*}.

\end{description}
\subsection*{OPTIONS\label{test_OPTIONS}\index{test!OPTIONS}}
\begin{description}

\item[\textbf{-h, -help}] \mbox{}

Print a brief help message and exit.


\item[\textbf{-m, -man}] \mbox{}

Print the manual page and exit.


\item[\textbf{-v, -verbose}] \mbox{}

Execute verbosely.


\item[\textbf{-Cpost}] \mbox{}

Call \texttt{pel run} with this option for all runs.


\item[\textbf{-Call}] \mbox{}

Call \texttt{pel run} with this option for all runs.


\item[\textbf{-build\_pattern} \emph{filename}] \mbox{}

Call \texttt{pel run} with this option for all runs.


\item[\textbf{-verify\_pattern} \emph{filename}] \mbox{}

Do not perform the runs, but instead use the PELICANS-based application
"check" (contained in the argument \emph{exe}) to check the conformity
of all reference data file \emph{data.pel} with the pattern file \emph{filename}.


\item[\textbf{-build\_then\_verify\_pattern} \emph{filename}] \mbox{}

Call \texttt{pel run} with the option \textbf{-build\_pattern} \emph{filename} 
for all runs and then check the conformity
of all reference data file \emph{data.pel} with the created pattern file
\emph{filename} (equivalent to two calls of \texttt{pel test}
with successively the \textbf{-build\_pattern} and the \textbf{-verify\_pattern} options).


\item[\textbf{-test\_directory} \emph{dirname}] \mbox{}

Duplicate the hierarchy of directories containing the reference runs
in the subdirectory \emph{dirname} of the working directory,
and run the application in the subdirectories of \emph{dirname} for further
result comparison with the reference runs (default: \emph{PELICANS\_TEST}).


\item[\textbf{-peltest\_exe} \emph{texe}] \mbox{}

Specify the executable containing the "peltest" and "pelcmp" applications.
Default is the argument \emph{exe} itself.


\item[\textbf{-dbl\_eps} \emph{eps}] \mbox{}

Specify the a\_dbl\_eps argument in calls to PEL::double\_equality
when comparing floating point values. This option is only
significant for files with format PEL, CSV or GENE.


\item[\textbf{-dbl\_min} \emph{min}] \mbox{}

Specify the a\_dbl\_min argument in calls to PEL::double\_equality
when comparing floating point values. This option is only
significant for files with format PEL, CSV or GENE.


\item[\textbf{-exact}] \mbox{}

Always perform comparisons between floating point values
without any tolerance, whatever settings of a\_dbl\_eps
and a\_dbl\_eps in files \emph{config.pel}.

\end{description}
\subsection*{EXAMPLES\label{test_EXAMPLES}\index{test!EXAMPLES}}
\begin{description}

\item[\texttt{pel test ../bin/exe ../RegressionTests}] \mbox{}

Run executable \emph{exe} located in the directory \emph{../bin}
with all data files \emph{data.pel} contained in the subdirectories
of \emph{../RegressionTests} and compare the results with the reference ones.
Create a report file in the current directory recording the conclusions
of all comparisons.


\item[\texttt{pel test -build\_pattern pat.pel ../bin/exe ../Tests}] \mbox{}

Same as before, with in addition the learning and storage of the requested
structure of the data files in \emph{pat.pel}.


\item[\texttt{pel test -verify\_pattern pat.pel ../bin/exe ../Appli}] \mbox{}

Check the conformance with \emph{pat.pel} of all files \emph{data.pel} contained 
in a subdirectory of \emph{../Appli}, and record the conclusions in a report 
file in the current directory.

\end{description}
\clearpage
\section{time\label{time}\index{time}}


Management of a CPU time database for PELICANS-based applications

\subsection*{SYNOPSIS\label{time_SYNOPSIS}\index{time!SYNOPSIS}}


pel time [-help$|$-man]



pel time [-verbose] -build\_and\_add \emph{file.db}



pel time [-verbose] -merge \emph{target.db} \emph{source.db}



pel time [-verbose] -compare \emph{file1.db} \emph{file2.db}



pel time [-verbose] -print \emph{file.db}

\subsection*{DESCRIPTION\label{time_DESCRIPTION}\index{time!DESCRIPTION}}


\texttt{pel time} is devoted to the management of
data bases collecting the execution time of PELICANS-based applications.
This tool is used to manage execution timing database.
These databases are built from result files produced by PELICANS-based applications.

\subsection*{OPTIONS\label{time_OPTIONS}\index{time!OPTIONS}}
\begin{description}

\item[\textbf{-h, -help}] \mbox{}

Print a brief help message and exit.


\item[\textbf{-m, -man}] \mbox{}

Print the manual page and exit.


\item[\textbf{-v, -verbose}] \mbox{}

Execute verbosely.


\item[\textbf{-build\_and\_add} \emph{file.db}] \mbox{}

Search recursively from the current directory all
files called \emph{resu}, which are supposed to contain the messages
associated to the run of a PELICANS-based application (collected
with the \texttt{pel -run} utility). For each file \emph{resu}, the execution time
is read and added into \emph{file.db} with the directory name as key (entries
that already existed are replaced).


\item[\textbf{-merge} \emph{target.db} \emph{source.db}] \mbox{}

Add all entries of \emph{source.db} into \emph{target.db}
(the entries that already exist in \emph{target.db} are replaced).


\item[\textbf{-compare} \emph{file1.db} \emph{file2.db}] \mbox{}

Compare \emph{file1.db} and \emph{file2.db}two database and print 
the differences.


\item[\textbf{-print} \emph{file.db}] \mbox{}

Display the content of \emph{file.db}.

\end{description}
\subsection*{EXAMPLES\label{time_EXAMPLES}\index{time!EXAMPLES}}
\begin{description}

\item[\texttt{pel time -build\_and\_add cpu.db}] \mbox{}

Build the database \emph{cpu.db} from all the \emph{resu} files of the
subdirectories of the current directory.


\item[\texttt{pel time -print cpu.db}] \mbox{}

Display the content of \emph{cpu.db}


\item[\texttt{pel time -compare cur.db ref.db}] \mbox{}

Compare \emph{cur.db} and \emph{ref.db}

\end{description}
\clearpage
\section{arch\label{arch}\index{arch}}


Discover and name the compiler architecture.

\subsection*{SYNOPSIS\label{arch_SYNOPSIS}\index{arch!SYNOPSIS}}


pel arch [-help$|$-man]



pel arch [-verbose] compiler



pel arch -getvariable $<$var$>$ compiler



pel arch -getvariable\_extra $<$var$>$ compiler

\subsection*{DESCRIPTION\label{arch_DESCRIPTION}\index{arch!DESCRIPTION}}


\texttt{pel arch} discovers the compiler architecture by selecting the
architecture-makefile and the extra-makefile that are the more closely
related to the current machine, the chosen compiler and external APIs, and
returns a string caracterizing the matching compiler architecture (its name).



\texttt{pel arch} searches successively two files:

\begin{itemize}

\item 

the architecture-makefile, called:  \emph{xxx.mak},  which essentially formalizes
the usage of the current compiler on the current machine;


\item 

the extra-makefile, called \emph{extra-xxx.mak}, which essentially describes the
linkage of the enabled external APIs with PELICANS on the current machine.

\end{itemize}


In both cases, \emph{xxx} denotes symbolically a character sequence which matches
one of the following patterns tried out in sequence:

\begin{enumerate}

\item 1.

$<$hostname$>$-$<$compiler$>$


\item 2.

$<$hostname$>$


\item 3.

$<$sysname$>$-$<$release$>$-$<$compiler$>$


\item 4.

$<$sysname$>$-$<$compiler$>$


\item 5.

$<$compiler$>$


\item 6.

$<$sysname$>$-$<$release$>$


\item 7.

$<$sysname$>$

\end{enumerate}


Where :

\begin{description}

\item[hostname] \mbox{}

is the name of the current host. It may be substitued if a file named
'arch\_file.cfg' exists in the searched paths.


\item[sysname] \mbox{}

is the name of the curent operating system name given by uname(1).


\item[release] \mbox{}

is the release of the curent operating system name given by uname(1).

\end{description}
\subsubsection*{The Searched Paths\label{arch_The_Searched_Paths}\index{arch!The Searched Paths}}


First, \texttt{pel arch} searches in the directory given by the environment
variable \emph{PELARCHDIR} (if defined), then in the \emph{\$PELICANSHOME/etc}
directory (if it is not defined, the subdirectory
\texttt{etc} of the current directory is searched instead).

\subsubsection*{Hostname Substitution\label{arch_Hostname_Substitution}\index{arch!Hostname Substitution}}


When a file named 'arch\_file.cfg' is encountered in the searched paths,
\texttt{pel arch} tries to substitute the current hostname by an alias name found
in this file. The first match found returns. When no match is found, the
current hostname is used.



This file is a two columns file. Comments starts with '\#' (sharp).

\begin{description}

\item[Column 1 :] \mbox{}

contains a perl (perlre(1)) regular expression matching hostnames.


\item[Column 2 :] \mbox{}

contains the alias for the regular expression.

\end{description}


'arch\_file.cfg' file example:
 sinux1 pinux    \# sinux1 vers pinux
 sinux$\backslash$d+ sinux  \# les autres noeuds sinux vers sinux
 pinux$\backslash$d+ pinux  \# les noeuds pinux vers pinux

\subsection*{OPTIONS\label{arch_OPTIONS}\index{arch!OPTIONS}}
\begin{description}

\item[\textbf{-h, -help}] \mbox{}

Print a brief help message and exit.


\item[\textbf{-m, -man}] \mbox{}

Print the manual page and exit.


\item[\textbf{-v, -verbose}] \mbox{}

Execute verbosely.


\item[\textbf{-getvariable} var] \mbox{}

Return the value of the variable var
defined in the achitecture-makefile of the current
compiler architecture.


\item[\textbf{-getvariable\_extra} var] \mbox{}

Return the value of the variable var
defined in the extra-makefile of the current
compiler architecture.

\end{description}
\subsection*{ARGUMENTS\label{arch_ARGUMENTS}\index{arch!ARGUMENTS}}
\begin{description}

\item[\textbf{compiler}] \mbox{}

Name of the compiler for which a compiler architecture
for the current hardware platform
will be searched.

\end{description}
\subsection*{EXAMPLE\label{arch_EXAMPLE}\index{arch!EXAMPLE}}
\begin{description}

\item[\texttt{pel arch -verbose CC}] \mbox{}

Find the available compiler architecture for compiler CC
and return a string caracterizing it.


\item[\texttt{pel arch -getvariable DYNAMIC\_LIB\_EXT CC}] \mbox{}

Find the available compiler architecture for compiler CC
and return the value of the variable DYNAMIC\_LIB\_EXT that it defines
in the achitecture-makefile.

\end{description}
\subsection*{ENVIRONMENT\label{arch_ENVIRONMENT}\index{arch!ENVIRONMENT}}
\begin{description}

\item[PELICANSHOME] \mbox{}

The PELICANS root directory.


\item[PELARCHDIR] \mbox{}

A user directory where user's architecture GNU Makefiles are stored.

\end{description}
\clearpage
\section{predoc\label{predoc}\index{predoc}}


Possible preparation before using "peldoc"

\subsection*{SYNOPSIS\label{predoc_SYNOPSIS}\index{predoc!SYNOPSIS}}


pel predoc [-help$|$-man]



pel predoc [options...] \emph{descfile} \emph{datfile} appli \emph{dirs}

\subsection*{DESCRIPTION\label{predoc_DESCRIPTION}\index{predoc!DESCRIPTION}}


\texttt{pel predoc} will traverse a list of directory hierarchies
from which it will infer a particular
packaging and create a data file, both
preparing a subsequent use of the PELICANS-based application \texttt{peldoc}.

\subsection*{ARGUMENTS\label{predoc_ARGUMENTS}\index{predoc!ARGUMENTS}}
\begin{description}

\item[\textbf{ \emph{descfile} }] \mbox{}

Name of the description file to be produced.


\item[\textbf{ \emph{datfile} }] \mbox{}

Name of the \texttt{peldoc} data file to be produced.


\item[\textbf{appli}] \mbox{}

Name given by \texttt{peldoc} for the application to be documented.


\item[\textbf{ \emph{dirs} }] \mbox{}

List of the directories from which the packaging of the
classes will be inferred.

\end{description}
\subsection*{OPTIONS\label{predoc_OPTIONS}\index{predoc!OPTIONS}}
\begin{description}

\item[\textbf{-h, -help}] \mbox{}

Print a brief help message and exit.


\item[\textbf{-m, -man}] \mbox{}

Print the manual page and exit.


\item[\textbf{-v, -verbose}] \mbox{}

Execute verbosely.


\item[\textbf{-Wno\_unresolved}] \mbox{}

Add options to the data file of \texttt{peldoc} so that it will inhibit
warning messages for assertions expressed with functions that are
implemented outside the current application.

\end{description}
\subsection*{EXAMPLE\label{predoc_EXAMPLE}\index{predoc!EXAMPLE}}
\begin{description}

\item[\texttt{pel predoc doc/description.txt doc/data.pel beauty .}] \mbox{}

A description file \emph{description.txt} and a \texttt{peldoc} data file
\emph{data.pel} will be produced  in the subdirectory
\emph{doc} of the current directory, for an application that will be
called "beauty", by recursively scanning all subdirectories
of the current directory.

\end{description}
\clearpage
\section{newclass\label{newclass}\index{newclass}}


Text files creation for a new class

\subsection*{SYNOPSIS\label{newclass_SYNOPSIS}\index{newclass!SYNOPSIS}}


pel newclass [-help$|$-man]



pel newclass [-path \emph{searchdir}] model newclass



pel newclass [-pelicans] model newclass

\subsection*{DESCRIPTION\label{newclass_DESCRIPTION}\index{newclass!DESCRIPTION}}


\texttt{pel newclass} creates the text files associated to
a new class by copying those of a model class
and replacing the name of the model class by the name
of the new class.



The text files associated to the model class are assumed
to have one of the following extensions : ".hh", ".icc" or ".cc" .
They are searched in a set of paths defined by the calling
options, or in the current directory if none.

\subsection*{OPTIONS\label{newclass_OPTIONS}\index{newclass!OPTIONS}}
\begin{description}

\item[\textbf{-h, -help}] \mbox{}

Print a brief help message and exit.


\item[\textbf{-m, -man}] \mbox{}

Print the manual page and exit.


\item[\textbf{-v, -verbose}] \mbox{}

Execute verbosely.


\item[\textbf{-path} \emph{searchdir}] \mbox{}

Add directory \emph{searchdir} to the list of paths that
\texttt{pel newclass} will search for files
\emph{model.hh}, \emph{model.cc} or \emph{model.icc} .
This option may be used any number of times.


\item[\textbf{-pelicans}] \mbox{}

Add all directories of the PELICANS repository
defined by the environment variable \texttt{PELICANSHOME}
to the list of paths that \texttt{pel newclass} will search for files
\emph{model.hh}, \emph{model.cc} or \emph{model.icc} .

\end{description}
\subsection*{ARGUMENTS\label{newclass_ARGUMENTS}\index{newclass!ARGUMENTS}}
\begin{description}

\item[\textbf{model}] \mbox{}

Name of the model class. The files \emph{model.hh}, \emph{model.cc} and \emph{model.icc}
will be searched, copied in the current directory if found,
and any occurence of \texttt{model} will be replaced by \texttt{newclass}.


\item[\textbf{newclass}] \mbox{}

Name of the new class. The produced files will be \emph{newclass.hh},
\emph{newclass.icc} and \emph{newclass.cc}, prodived that related files
for the model class were found.

\end{description}
\subsection*{EXAMPLES\label{newclass_EXAMPLES}\index{newclass!EXAMPLES}}
\begin{verbatim}
 pel newclass -pelicans GE_QuadratureRule_TEST MY_Class_TEST>
\end{verbatim}
\begin{verbatim}
 pel   newclass -path ../src Model NewClass>
\end{verbatim}
\clearpage
\section{cmp\label{cmp}\index{cmp}}


Requests forward to the PELICANS-based application "pelcmp"

\subsection*{SYNOPSIS\label{cmp_SYNOPSIS}\index{cmp!SYNOPSIS}}


pel cmp [-help$|$-man]



pel cmp [options...] \emph{file1} \emph{file2}

\subsection*{DESCRIPTION\label{cmp_DESCRIPTION}\index{cmp!DESCRIPTION}}


The PELICANS-based application "pelcmp" is called with the given
options and arguments.

\subsection*{OPTIONS\label{cmp_OPTIONS}\index{cmp!OPTIONS}}
\begin{description}

\item[\textbf{-h, -help}] \mbox{}

Print a brief help message and exit.


\item[\textbf{-m, -man}] \mbox{}

Print the manual page and exit.


\item[\textbf{-v, -verbose}] \mbox{}

Execute verbosely.


\item[\textbf{-m, -modules} \texttt{mod}] \mbox{}

Limit the comparison to \texttt{mod}.


\item[\textbf{-d, -data} \texttt{datum}] \mbox{}

Limit the comparison to \texttt{datum}

\end{description}
\subsection*{ARGUMENTS\label{cmp_ARGUMENTS}\index{cmp!ARGUMENTS}}
\begin{description}

\item[\emph{file1}, \emph{file2}] \mbox{}

Name of the files containing containing the 
data to be compared.

\end{description}
\subsection*{EXAMPLE\label{cmp_EXAMPLE}\index{cmp!EXAMPLE}}
\begin{description}

\item[\texttt{pel cmp file\_ref.pel file.pel}] \mbox{}

Compare the data of the files \emph{file\_ref.pel} and \emph{file.pel}.

\end{description}
\clearpage
\section{cvgce\label{cvgce}\index{cvgce}}


Extraction of error norms from \emph{save\_*.gene} text files

\subsection*{SYNOPSIS\label{cvgce_SYNOPSIS}\index{cvgce!SYNOPSIS}}


pel cvgce [-help$|$-man]



pel cvgce [-v] $<$norm\_saving\_names$>$

\subsection*{DESCRIPTION\label{cvgce_DESCRIPTION}\index{cvgce!DESCRIPTION}}


The convergence properties of numerical schemes when refining
the time step and the mesh size may be tested using the classes
\emph{FE\_Launcher} and \emph{FE\_ComparatorWithAnalytic}. In this context and
when the calculation
results are saved with \emph{PEL\_TICwriter} (in the "text" mode), a sequence
of \emph{save\_*.gene} files are generated.
Each \emph{save\_*.gene} corresponds to a given
time step, to a given mesh size (specified on creation of \emph{FE\_Launcher}),
and stores at all cycles the error norms that where requested
from \emph{FE\_ComparatorWithAnalytic}.



Given such a sequence of \emph{save\_*.gene} files, \texttt{pel cvgce} extracts
all the values of the error norms, reorganizes these data to stress the
dependencies with the time step and the mesh size (for a given mesh size,
dependence with the time step ; for a given time step, dependence with
the mesh size), and finally produces multicolumn ascii files that may be
used for graphic postprocessing by usual tools (eg gnuplot or xmgrace).

\subsection*{OPTIONS\label{cvgce_OPTIONS}\index{cvgce!OPTIONS}}
\begin{description}

\item[\textbf{-h, -help}] \mbox{}

Print a brief help message and exit.


\item[\textbf{-m, -man}] \mbox{}

Print the manual page and exit.


\item[\textbf{-v, -verbose}] \mbox{}

Execute verbosely (\textbf{recommended}).

\end{description}
\subsection*{ARGUMENTS\label{cvgce_ARGUMENTS}\index{cvgce!ARGUMENTS}}
\begin{description}

\item[\textbf{norm\_saving\_names}] \mbox{}

Items occuring in the entry of keyword \texttt{norm\_saving\_names}
in the Hierarchical Data Structure used to create the
\emph{FE\_ComparatorWithAnalytic} object.

\end{description}
\subsection*{EXAMPLE\label{cvgce_EXAMPLE}\index{cvgce!EXAMPLE}}
\begin{description}

\item[\texttt{pel cvgce -v XLDU XHDU}] \mbox{}

Extract from all the \emph{save\_*.gene} ascii files of the current directory
the entries \texttt{XLDU} and \texttt{XHDU}, and produce the following files:



\emph{XLDU\_h.txt} : for all time steps, XLDU as a function of the mesh size



\emph{XLDU\_dt.txt} : for all mesh sizes, XLDU as a function of the time step



\emph{XHDU\_h.txt} : for all time steps, XHDU as a function of the mesh size



\emph{XHDU\_dt.txt} : for all mesh sizes, XHDU as a function of the time step



The \emph{save\_*.gene} files have been produced with an application using
a \emph{FE\_ComparatorWithAnalytic} which has been created using a PELICANS
Hierarchical Data Structure. This latter must be such that \texttt{"XLDU"}
and \texttt{"XLDU"} occur in its entry of keyword \texttt{norm\_saving\_names}.

\end{description}
\printindex

\end{document}
